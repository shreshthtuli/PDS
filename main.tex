\documentclass{article}

\usepackage[utf8]{inputenc}
\usepackage{hyperref}

\title{PDS Model(COP290-Design Practices)}

\author{Sankalan Pal Chowdhury, Shreshth Tuli }

\date{April 2018}



\begin{document}



\maketitle

\section{Background}
India's Public Distribution System (PDS) is the largest distribution network of its kind in the world.  PDS was introduced around World War II as a war-time rationing measure.  Before the 1960s, distribution through PDS was generally dependent on imports of food grains.  It was expanded in the 1960s as a response to the food shortages of the time; subsequently, the government set up the Agriculture Prices Commission and the Food Corporation of India  to improve domestic procurement and storage of food grains for PDS.  By the 1970s, PDS had evolved into a universal scheme for the distribution of subsidized food.  In the 1990s, the scheme was revamped to improve access of food grains to people in hilly and inaccessible areas, and to target the poor.   Subsequently, in 1997, the government launched the Targeted Public Distribution System (TPDS), with a focus on the poor. \\ \\ TPDS aims to provide subsidized food and fuel to the poor through a network of ration shops.  Food grains such as rice and wheat that are provided under TPDS are procured from farmers, allocated to states and delivered to the ration shop where the beneficiary buys his entitlement.  The center and states share the responsibilities of identifying the poor, procuring grains and delivering food grains to beneficiaries. \\ \\ In September 2013, Parliament enacted the National Food Security Act, 2013.  The Act relies largely on the existing TPDS to deliver food grains as legal entitlements to poor households.  This marks a shift by making the right to food a justiciable right.  In order to understand the implications of this Act, the note maps the food supply chain from the farmer to the beneficiary, identifies challenges to implementation of TPDS, and discusses alternatives to reform TPDS.  It also details state-wise variations in the implementation of TPDS and discusses changes to the existing system by the Act. 

\section{Motivation}

As there are many ration shops and the customers coming to buy from ration shops are normally believed to be below poverty line and illiterate, the customers are fooled to a large extent. There are complaints related to the quality of the product they receive, the quantity they receive is many a times less than the quantity demanded by them as the employees steal from it. Moreover, they end up paying more for the quantity they receive. Also the quantity which is added in the ration card is wrong. So they cannot buy more the next time they need. So there is a lot of cheating and fooling of the customers that takes place.\\ \\ In lieu of these problems, I propose some suggestive measures and improvements in the form of a succinct functional model. Prior to model description it is important that we analyze the current model and build a list of requirements and specifications. 

\section{Demerits of current system}

The current system faces many issues both in the design as well as implementation. Broadly these issues can be categorized into the following major domains:

\begin{enumerate}
\item CORRUPTION AND LEAKAGE OF FOOD GRAINS IN PDS : \\
Corruption is a major problem in developing countries like India. According to studies, in India about 40 percent of food grains channeled through public distribution system is diverted to the open market. Even though, the corruption exists in all the stages of system, it is clearly noticeable at fair price shops.
\item QUALITY AND QUANTITY : \\
Public distribution system in country suffers from irregular and poor quality of entitlements that distributed through fair price shops. The poor quality of food grains in PDS is evidence of public sector’s inefficiency in the state of Jammu and Kashmir and other states. Adulteration, quality and underweight is a major problem faced by the beneficiaries.
\item BOGUS CARDS : \\
The presence of bogus ration cards in the system makes significant challenges, the bogus cards are the cards that are issued for fictitious family and genuine ration cards are used by someone else. The actual entitlements that are meant for poor households are taken away by bogus cards. The bogus cards are mainly created by FPS owners for the purpose of making money through diverting entitlements to open market, it is obvious that the commission given by government is low for distribution of entitlements through FPS and people are willing to pay large amount of money illegally and illegally to obtain the dealership, because they can obtain huge money by diverting entitlements to open market using bogus cards.
\item WRONG CLASSIFICATION OF ECONOMIC STATUS : \\
Government of India classifies the households based on their socio-economic status as above poverty line (APL), below poverty line (BPL) and Anthyodaya Anna Yojana (AAY) to distribute the PDS entitlements. Khera R, 2011 indicates, among the 400 studied samples 25 percent of households were wrongly classified as BPL and 44 percent of real BPL were classified as other classes in Rajasthan.
\item NON-AVAILABILITY OF GRAINS AND IRREGULAR FUNCTION OF FPS : \\
It has been witnessed by several scholars that, entitlements are not available in the FPS as well as the FPS does not function on right time.
\item LEAKAGE OF FOOD GRAINS : \\
TPDS suffers from large leakages of food grains during transportation to and from ration shops into the open market.  In an evaluation of TPDS, the Planning Commission found 36\% leakage of PDS rice and wheat at the all-India level.
\end{enumerate}
Apart from the noted problems above, other problems that discussed by scholars are inadequate of physical access to FPS, rural and urban bias, regional variability, poor economic condition of households, unawareness of beneficiaries and FPS owners about PDS, mortgaging of ration cards and logistics.

\section{Specifications}

Laying out specifications for the problem is not a difficult task but becomes non trivial when many factors like feasibility, remote access, tamper proof, etc. are considered. After considerable thought the PDS system's specifications and the problem at hand of improvising the current solution considering the political and socio-economic factors; include but are not limited to:

\begin{enumerate}
\item Timely and need based allocation of ration food
\item Secured and non tamper able framework of resource delivery 
\item Prevention of diversion of essential commodities. 
\item Containment of arbitrary decision making at all levels. 
\item Induction of transparency and accountability in operations
\item Reduction of redundant workload of department employees. 
\item Security and control of confidential data. 
\item Fast disposal of stakeholder grievances
\item Dissemination of information as per public requirements
\item Protecting the interest of all the stakeholders. 
\item To improve service delivery and create transparency 
\item To empower beneficiary 
\item To weed out bad FPS and bogus Ration Cards. 
\item Ensuring unadulterated and right amounts of food reach the beneficiaries
\end{enumerate}

\section{Proposed solution}

\subsection{Assumptions}

\subsection{Relaxations}

\subsection{Design decisions}
 
\subsection{Functional model}

\subsection{Implementation Constraints}

\section{Further scope}

\section{Conclusions}

\begin{thebibliography}{99}

\bibitem{c1} Planning Commission : \href{http://planningcommission.nic.in/plans/planrel/fiveyr/10th/volume2/v2_ch3_4.pdf}{website}
\bibitem{c2} \href{http://www.prsindia.org/administrator/uploads/general/1388728622~~TPDS%20Thematic%20Note.pdf}{PRS India}
\bibitem{c3} \href{http://www.kscst.iisc.ernet.in/spp/41_series/40S_bestprojreports/40S_BE_0366.pdf}{KSTS}
\bibitem{c4} \href{https://ac.els-cdn.com/S1877050915003324/1-s2.0-S1877050915003324-main.pdf?_tid=9ac7b378-d73e-435a-8ff6-14051bf3e25c&acdnat=1524581743_16e91753752ba258816d70c10d11c6c3}{Automated Ration Distribution System : Science Direct}
\bibitem{c5} \href{https://geleservices.com/GEL/Download/RFP-POS%20.pdf}{POS Based distribution system : Elsevier}
\bibitem{c6} \href{http://www.epw.in/journal/2017/50/special-articles/aadhaar-and-food-security-jharkhand.html}{Aadhaar and food security in Jharkhand : Economics and Political Weekly}
\bibitem{c7} Khera R, (2011) India's Public Distribution System: Utilisation and Impact. The Journal of Development Studies 47 (7), 1038-1060
\bibitem{c8} \href{http://www.ijsrd.com/articles/IJSRDV4I90439.pdf}{Major Drawbacks of PDS : .Mahalingam, Akash Raj D}


\end{thebibliography}





\end{document}